\documentclass[a4paper]{article}

\usepackage{graphicx}
\usepackage{parskip}
\usepackage{tcolorbox}
\usepackage[margin=0.5in]{geometry}
\pagenumbering{gobble}

\newcommand{\formel}[3][]{
	\textbf{#2:}\\ #3
	\ifx #1\undefined \else
	{\tiny(#1)}
	\fi
}

\begin{document}

\begin{center}
	\resizebox{\columnwidth}{!}{%
		\begin{tabular}{|c|c|c|c|c|c|c|c|c|c|c|c|c|}
			\hline
			tera (T) & giga (G) & mega (M) & kilo (k) & hekto (h) & deka (da) & deci (d) & centi (c) & milli (m) & mikro (µ) & nano (n) & piko (p) & femto (f) \\
			$10^{12}$ & $10^9$ & $10^6$ & $10^3$ & $10^2$ & $10^1$ & $10^{-1}$ & $10^{-2}$ & $10^{-3}$ & $10^{-6}$ & $10^{-9}$ & $10^{-12}$ & $10^{-15}$ \\
			\hline
		\end{tabular}
		}
\end{center}

\formel[Comment]{Densitet}
{$\rho = \frac{m}{v}$}

\formel[Enheten joule, J]{Arbete}
{$ W=F \cdot s = mas $}

\begin{tcolorbox}[title=Konjugat- och kvadreringsreglerna]
	$(a+b)^2=a^2+2ab+b^2$

	$(a-b)^2=a^2-2ab+b^2$

	$(a+b)(a-b)=a^2-b^2$
\end{tcolorbox}

\end{document}
